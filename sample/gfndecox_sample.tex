\documentclass{jsarticle}
\makeatletter
  \usepackage{lmodern}
  \usepackage[T1]{fontenc}
  \usepackage[dvipdfmx]{graphicx}
  \usepackage[dvipdfmx]{xcolor}
  \usepackage{tikz}
  \usepackage{ascmac}
% ---- gfncls ----
  \usepackage{gfncls}
  \setsectiontheme{line}
    %(現状はまだ line スタイルしかありません)
% ---- gfndecox ----
  \usepackage{gfndecox}
  \newdecoclass{\defpar}{定義}{ul}{subsubsection}
  \newnonumberdecoclass{\example}{例}{top}
  \newdecoclass{\thmpar}{定理}{box}{subsubsection}
  \newnonumberdecoclass{\plainpar}{}{top}
% ---- 体裁 ----
  \setlength{\parskip}{1em}
% ---- 意味マークアップ ----
  \def\newword#1{\textsf{#1}}
  \def\pkgname#1{\texttt{#1}}
  \def\bksl{\symbol{"5C}}
  \def\meta#1{\hskip1ex\relax\textrm{#1}\hskip1ex\relax}
% ---- ----
  \title{\pkgname{gfncls},\pkgname{gfndecox}パッケージ使用例}
  \author{gfn}
\makeatother
\begin{document}
  \maketitle
  \section{おでん理論}
    \subsection{おでん}
      \plainpar{はじめに}{
        この文書に掲げた例と説明は大変杜撰である.
      }
      \defpar{ちくわ}{
        おでんの具のうち,円環体と同相であるものを\newword{広義ちくわ}と呼ぶ.
      }
      \example{}{狭義ちくわおよびちくわ麸は広義ちくわである.}
      \thmpar{自明な定理}{
        \TeX と\LaTeX を混同することはアレである.
      }
  \section{本題}
    \subsection{章・節体裁}
      \plainpar{}{
        \pkgname{gfncls}パッケージは
        (本当はパッケージが担うべき役目とはいえないのですが)
        クラスファイルのごく一部を上書きするパッケージです.
        \begin{center}
          \texttt{\bksl setsectiontheme\{\meta{体裁指定}\}}
        \end{center}
        で章・節の体裁を指定できますが,
        現状ではまだ\texttt{std}オプションと\texttt{line}オプションしかありません.
        この文書の体裁は\texttt{line}オプションです.
        \texttt{line}オプションは\pkgname{tikz}パッケージに依存するので別個に読み込んでください.
      }
    \subsection{段落分類の宣言}
      \defpar{段落分類}{
        段落に分類を与え,分類ごとに装飾を施すのが\pkgname{gfndecox}パッケージの主要な仕事です.
        この分類を\newword{段落分類}と呼ぶことにします.
        段落分類は,基本的に
        \begin{center}
          \texttt{\bksl newdecoclass\{\meta{制御綴}\}\{\meta{分類名}\}\{\meta{体裁指定}\}\{\meta{\LaTeX 式カウンタ}\}}
        \end{center}
        という形で宣言します.分類名とカウンタによる番号がタイトルとなり,
        段落全体に体裁指定にしたがった装飾が施されます(
        といっても体裁指定は現状では\texttt{std},\texttt{ul},\texttt{top},
        \texttt{box}くらいしかありません).
        使うときには
        \begin{center}
          \texttt{\meta{制御綴}\{\meta{タイトル}\}\{\meta{本文}\}}
        \end{center}
        とします.
        番号を振りたくない箇所には\texttt{*}をつけることで一部だけ消すことができ,
        また番号を参照したい場合は\texttt{\bksl label}命令ではなくオプション引数で指定できます.
      \decosep
        根本的に番号のいらない段落分類は
        \begin{center}
          \texttt{\bksl newnonumberdecoclass\{\meta{制御綴}\}\{\meta{分類名}\}\{\meta{体裁指定}\}}
        \end{center}
        で宣言します.これによって宣言された段落分類は\texttt{*}をつけなくても番号が振られません.
        というより,\texttt{*}やオプション引数には対応していないので,
        つけるとエラーになるか,或いは意図されるような挙動はしません.
      }
    \subsection{使用例}
      \defpar*{番号をつけない}{
        \texttt{*}をつけると,番号がふられなくなります.
      }
      \defpar[def:reference]{相互参照}{
        相互参照はオプション引数で指定します.
      }
      \thmpar{引用するところの例}{
        「定義\ref{def:reference}によると,」……という具合に引用されます.
      }
      \defpar{段落分類中の改段落}{
        ひとつの分類中で改段落したい場合は,改段落したい箇所に\texttt{\bksl decosep}を入れてください.
        あまり短いと例が綺麗に見えなさそうなので,あとは無意味な文章で分量を水増ししておきましょう.
        そろそろ改段落したいなーいやしたくないかなーやっぱりしたいなーよし改段落するぞ
      \decosep
        改段落しました.まあでも後ろの段落も長くないと綺麗に見えませんよね,
        色は匂へど散りぬるを,吾が世誰ぞ恒ならむ,有為の奥山今日越えて,浅き夢見じ酔ひもせず.
      }
    \subsection{体裁一覧}
      \newdecoclass{\testpar}{表示例}{std}{subsubsection}
      \testpar{\texttt{std}オプション}{
        普通のタイトルつきの段落です.例によって残りは意味のない文章です:\quad
        色は匂へど散りぬるを,吾が世誰ぞ恒ならむ,有為の奥山今日越えて,浅き夢見じ酔ひもせず.
      }
      \newdecoclass{\testpar}{表示例}{ul}{subsubsection}
      \testpar{\texttt{ul}オプション}{
        タイトルに下線がつきます.やはり残りは意味のない文章です:\quad
        色は匂へど散りぬるを,吾が世誰ぞ恒ならむ,有為の奥山今日越えて,浅き夢見じ酔ひもせず.
      }
      \newdecoclass{\testpar}{表示例}{top}{subsubsection}
      \testpar{\texttt{top}オプション}{
        本文がタイトルと同一の行で始まります.
        色は匂へど散りぬるを,吾が世誰ぞ恒ならむ,有為の奥山今日越えて,浅き夢見じ酔ひもせず.
      }
      \newdecoclass{\testpar}{表示例}{box}{subsubsection}
      \testpar{\texttt{box}オプション}{
        本文が箱で括られます.
        \pkgname{ascmac}パッケージに依存するので,別途読み込んでください.
        また,途中で改ページすることはできないので注意してください.
      }
\end{document}
